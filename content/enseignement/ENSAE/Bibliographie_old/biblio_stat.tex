\documentclass[a4paper,12pt]{article}

%%%%%%%%%%%%%%%%%%%%%%%%%%%%%%%%%%%%%%%%%%%%%%%%%%%%%%%%%%%
%%%%% 	pour le fran�ais et les accents 	      %%%%%
%%%%%%%%%%%%%%%%%%%%%%%%%%%%%%%%%%%%%%%%%%%%%%%%%%%%%%%%%%%

\usepackage[french]{babel}
%%%%%%%%%%%%%%%%%%%%%%%%%%%%%%%%%%%%%%%%%%%%%%%%%%%%%%%%%%%
\usepackage[T1]{fontenc}
\usepackage{amsmath}
\usepackage{amssymb}

\usepackage{geometry}
\usepackage{amsthm}       % for theorem definitions
\usepackage{dsfont}       % for  \mathds and nice 0/1 functions
\usepackage{graphicx}     % for images and graphics
\usepackage{stmaryrd}     % for \llbrackets
\usepackage{color}
\usepackage{subeqnarray}
\usepackage{subfigure}

%%%%%%%%%%%%%%%%%%%%%%%%%%%%%%%%%%%%%%%%%%%%%%%%%%%%%%%%%%%
%%%%% 	package pour les algorithmes		       %%%%
%%%%%%%%%%%%%%%%%%%%%%%%%%%%%%%%%%%%%%%%%%%%%%%%%%%%%%%%%%%



\usepackage{algpseudocode}
\usepackage{algorithm}
\usepackage{algpascal}

%%%%%%%%%%%%%%%%%%%%%%%%%%%%%%%%%%%%%%%%%%%%%%%%%%%%%%%%%%%
%%%%% 	pour des liens hypertextes  et du jpg  	      %%%%%
%%%%%%%%%%%%%%%%%%%%%%%%%%%%%%%%%%%%%%%%%%%%%%%%%%%%%%%%%%%




\usepackage[pdftex,a4paper,linkcolor=test,citecolor=vertsombre,colorlinks=true,bookmarks=true, plainpages=true,urlcolor=freeblue]{hyperref}
\usepackage{natbib}
\usepackage{bibtopic}


%%%%%%%%%%%%%%%%%%%%%%%%%%%%%%%%%%%%%%%%%%%%%%%%%%%%%%%%%%%
%%%%%  		Couleurs			%%%%%%%%%%%
%%%%%%%%%%%%%%%%%%%%%%%%%%%%%%%%%%%%%%%%%%%%%%%%%%%%%%%%%%%

\definecolor{test}{rgb}{0.64,0.16,0.16}
\definecolor{vertsombre}{rgb}{0.00,0.57,0.1}
\definecolor{freeblue}{rgb}{0.25,0.41,0.88}

%%%%%%%%%%%%%%%%%%%%%%%%%%%%%%%%%%%%%%%%%%%%%%%%%%%%%%%%%%%




%%%%%%%%%%%%%%%%%%%%%%%%%%%%%%%%%%%%%%%%%%%%%%%%%%%%%%%%%%%
%%%%%%%   	 SHORT CUTS:	  		 %%%%%%%%%%
%%%%%%%%%%%%%%%%%%%%%%%%%%%%%%%%%%%%%%%%%%%%%%%%%%%%%%%%%%%

\newcommand{\nc}{\newcommand}
\nc{\argmin}{\mathop{\mathrm{arg\,min}}}


%%%%%%%%%%%%%%%%%%%%%%%%%%%%%%%%%%%%%%%%%%%%%%%%%%%%%%%%%%%
%%%%%%%  		  SETS:			%%%%%%%%%%%
%%%%%%%%%%%%%%%%%%%%%%%%%%%%%%%%%%%%%%%%%%%%%%%%%%%%%%%%%%%

\def\R{\mathbb{R}}
\def\A{\mathbb{A}}
\def\B{\mathbb{B}}
\def\N{\mathbb{N}}
\def\L{\mathbb{L}}
\def\T{\mathbb{T}}
\def\G{\mathbb{G}}
\def\D{\mathbb{D}}
\def\H{\mathbb{H}}
\def\I{\mathbb{I}}
\def\F{\mathbb{F}}
\def\C{\mathbb{C}}
\def\Z{\mathbb{Z}}


%%%%%%%%%%%%%%%%%%%%%%%%%%%%%%%%%%%%%%%%%%%%%%%%%%%%%%%%%%%
%%%%%%%  		  PROBA			%%%%%%%%%%%
%%%%%%%%%%%%%%%%%%%%%%%%%%%%%%%%%%%%%%%%%%%%%%%%%%%%%%%%%%%

\nc{\ind}{\mathds{1}}
\def\P{\mathbb{P}}
\def\E{\mathbb{E}}
\def\V{\mathbb{V}\mbox{ar}\,}
\def\vt{\mbox{Var}_\theta\,}
\def\cov{\mbox{Cov}\,}
\def\ct{\mbox{Cov}_\theta\,}


%%%%%%%%%%%%%%%%%%%%%%%%%%%%%%%%%%%%%%%%%%%%%%%%%%%%%%%%%%%
%%%%%%%		classe de fonctions		%%%%%%%%%%%
%%%%%%%%%%%%%%%%%%%%%%%%%%%%%%%%%%%%%%%%%%%%%%%%%%%%%%%%%%%



\def\f{{\cal F}} 
\def\g{{\cal G}} 
\def\Lip{\mbox{Lip}\,}


%%%%%%%%%%%%%%%%%%%%%%%%%%%%%%%%%%%%%%%%%%%%%%%%%%%%%%%%%%%
%%%%%%  theorem and others stuffs in English	 %%%%%%%%%%
%%%%%%%%%%%%%%%%%%%%%%%%%%%%%%%%%%%%%%%%%%%%%%%%%%%%%%%%%%%
\newtheorem{theorem}{Th�or�me}[section]
\newtheorem{lemma}[theorem]{Lemme}
\newtheorem{ex}{Exemple}[section]
\newtheorem{corollary}[theorem]{Corollaire}
\newtheorem{prop}[theorem]{Proposition}
\theoremstyle{remark}
\theoremstyle{definition}
\newtheorem{definition}[theorem]{D�finition}
\newtheorem{rem}{Remarque}[section]


%%%%%%%%%%%%%%%%%%%%%%%%%%%%%%%%%%%%%%%%%%%%%%%%%%
%%%%%%		D�but du document 	    %%%%%%
%%%%%%%%%%%%%%%%%%%%%%%%%%%%%%%%%%%%%%%%%%%%%%%%%%



\begin{document}
\begin{center}
{\Huge\bfseries Quelques r�f�rences utiles pour avancer en Statistique} 

\par\bigskip
Joseph \textsc{Salmon} 
\par
\url{http://people.math.jussieu.fr/~salmon/}
\end{center}

%\tableofcontents
%\section{Introduction}





Voici une br�ve bibliographie de livres et d'autres supports (niveau M1/M2) en lien avec les statistiques. Attention, la plupart des r�f�rences sont en  anglais.


%%%%%%%%%%%%%%%%%%%%%%%%%%%%%%%%%%%%%%%%%%%%%%%%%%%%%%%%%%%%%%%%%%%%%%%%%%%%%%%%%%%%%%%%%%%%%%%%%%%%%%%%%%%%%%%%%%%%%%%%%%%%%%%%%%%%%%%%%%%%%%%%%%%%%%%%%%%%%%%%%%%%%%%%%%%%%%%%%%%%%%%%%%%%%%%%%%%%%%%%%%%%%%%%%%%%%%%%%%%%%%%%%%%%%%%%%%%%%%%%%%%%%%%%%%%%%%%%


\section{Pr�requis: Probabilit� et Int�gration}
\begin{btUnit}
\begin{btSect}[alpha]{refs}
Pour l'int�gration  \cite{Rudin87} existe en traduction francaise chez Dunod.
Pour les probabilit�s \cite{Cottrell_GenonCatalot_Duhamel_Meyre05}, \cite{Ouvrard08}, \cite{Ouvrard07} sont  en fran�ais,  \cite{Grimmett_Stirzaker01}.

\btPrintCited
\end{btSect}
\end{btUnit}

\section{Estimation et Test}
\begin{btUnit}
\begin{btSect}[alpha]{refs}

\nocite{Shao05}
\nocite{Casella_Berger90}
\nocite{Lehmann_Casella98} 
\nocite{Bickel_Docksum76}


\btPrintCited
\end{btSect}
\end{btUnit}




\section{Statistique non-param�trique}
\begin{btUnit}
\begin{btSect}[alpha]{refs}

Il est bon de commencer par \cite{Silverman86}, tr�s bien illustr� et facile � lire. Dans un second temps la th�orie est impeccable dans \cite{Tsybakov04}.
\btPrintCited
\end{btSect}
\end{btUnit}
% 
% 
% 






%%%%%%%%%%%%%%%%%%%%%%%%%%%%%%%%%%%%%%%%%%%%%%%%%%%%%%%%%%%%%%%%%%%%%%%%%%%%%%%%%%%%%%%%%%%%%%%%%%%%%%%%%%%%%%%%%%%%%%%%%%%%%%%%%%%%%%%%%%%%%%%%%%%%%%%%%%%%%%%%%%%%%%%%%%%%%%%%%%%%%%%%%%%%%%%%%%%%%%%%%%%%%%%%%%%%%%%%%%%%%%%%%%%%%%%%%%%%%%%%%%%%%%%%%%%%%%%%

% \section{Statistiques Bayesiennes}
% \begin{btUnit}
% \begin{btSect}[alpha]{refs}
% \nocite{Robert_Casella04}
% 
% \btPrintCited
% \end{btSect}
% \end{btUnit}


%%%%%%%%%%%%%%%%%%%%%%%%%%%%%%%%%%%%%%%%%%%%%%%%%%%%%%%%%%%%%%%%%%%%%%%%%%%%%%%%%%%%%%%%%%%%%%%%%%%%%%%%%%%%%%%%%%%%%%%%%%%%%%%%%%%%%%%%%%%%%%%%%%%%%%%%%%%%%%%%%%%%%%%%%%%%%%%%%%%%%%%%%%%%%%%%%%%%%%%%%%%%%%%%%%%%%%%%%%%%%%%%%%%%%%%%%%%%%%%%%%%%%%%%%%%%%%%%

\section{Ondelettes}
\begin{btUnit}
\begin{btSect}[alpha]{refs}
\cite{Mallat09} avec de jolies illustrations, est plut�t ax� th�orie du signal. \cite{Hardle_Kerkyacharian_Picard_Tsybakov98} aborde plut�t le versant statistiques. 
\btPrintCited
\end{btSect}
\end{btUnit}



%%%%%%%%%%%%%%%%%%%%%%%%%%%%%%%%%%%%%%%%%%%%%%%%%%%%%%%%%%%%%%%%%%%%%%%%%%%%%%%%%%%%%%%%%%%%%%%%%%%%%%%%%%%%%%%%%%%%%%%%%%%%%%%%%%%%%%%%%%%%%%%%%%%%%%%%%%%%%%%%%%%%%%%%%%%%%%%%%%%%%%%%%%%%%%%%%%%%%%%%%%%%%%%%%%%%%%%%%%%%%%%%%%%%%%%%%%%%%%%%%%%%%%%%%%%%%%%%




\section{Logiciels}


\begin{itemize}
\item[$\bullet$] Matlab  (payant): \url{http://www.mathworks.fr/} et de l'aide en traitement du signal \url{http://www.ceremade.dauphine.fr/~peyre/numerical-tour/}

\item[$\bullet$] Octave  (gratuit): (presque) Matlab compatible \url{http://www.gnu.org/software/octave/}

\item[$\bullet$] Scilab  (gratuit):  \url{http://www.scilab.org/}

\item[$\bullet$]
R (gratuit): \url{http://www.r-project.org/} et un bon tutorial d'Emmanuel Paradis pour d�buter \url{cran.r-project.org/doc/contrib/Paradis-rdebuts_fr.pdf}.
\end{itemize}


\end{document}